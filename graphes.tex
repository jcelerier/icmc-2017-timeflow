\documentclass[french,a4paper]{article}
\usepackage{microtype}
\usepackage{polyglossia}
\usepackage{geometry}
\setmainfont{Tribun ADF Std}
\begin{document}

\section{Graphe ``temporel'' / d'activation}
On définit le graphe dirigé acyclique $G_a=(V_a,E_a)$.
\begin{itemize}
    \item $V_a$ : points d'interaction.
    \item $E_a$ : écoulement entre les points (contraintes temporelles).
\end{itemize}

\section{Graphe hiérarchique}
Graphe de dépendances de données : une fonction a besoin de l'exécution d'une fonction précédente.

Optimisation : passer directement les valeurs. Mais en attendant, on fait du f1 -> store et read -> f2.

\section{Connection entre nœuds}
\begin{itemize}
\item Cas 1 : "optimiste" : on essaye de toujours envoyer les données
\item Cas 2 : "pessimiste" : on n'envoie les données que lorsque tous les nœuds du graphe sont actifs
\item Cas 3 : "retard" : les données sont bufferisées pour envoyer lorsqu'un objet subséquent devient actif. 
Note : que se passe-t-il si on a plusieurs utilisateurs subséquents ?
 Est-ce qu'on reprend du début à chaque fois (le pointeur est dans l'objet situé au bout de l'arête), ou de là ou on s'est arrêté de lire (le pointeur est dans l'objet situé à l'origine de l'arête).
\item Cas 4 : envoyer en parallèle ?
\item Cas 5 : ce qu'il y a actuellement : les nœuds suivants remplacent les résultats des noeuds précédents.
\end{itemize}

Pour pouvoir avoir les différents types de connections dans un seul scénario, il faut avoir moyen de définir des ensembles de processus qui vont pouvoir être liés entre eux.

Dans un groupe on peut définir : 
\begin{itemize}
    \item Une politique.
    \item Des liens qui manifestent la volonté de l'utilisateur.
\end{itemize}

Cas ou on a une variable interne au système : on est toujours soit dans le case 2 ou 3.

Cas ou l'utilisateur doit spécifier explicitement les relations de données : 3, 5

\section{Modélisation par des dataflow existants}
\subsection{Cas du retard}
On a juste besoin d'ajouter un nœud.

\subsection{Cas 1 : séquence opportuniste}
Les mettre en série. Idem, est-ce suffisant ? 

\subsection{Cas 2 : séquence pessimiste}

\subsection{Cas 4 : Parallèle (utilité?)}

\subsection{Cas 5 : remplacement}
Les mettre en parallèle, le dernier est celui qu'on veut remplacer (compter que des effets de bord peuvent arriver, donc il doit y avoir un ordre. Est-ce suffisant ? Ou bien faut-il tout voir dans un seul dataflow...)



** Réfléchir à des examples standard qu'on veut **


Spécification des sous-graphes

Définir l'interface utilisateur
\end{document}
