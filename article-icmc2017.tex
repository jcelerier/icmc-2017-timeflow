\documentclass[a4paper,twocolumns]{article}
\usepackage{microtype}
\usepackage{polyglossia}
\usepackage{geometry}
\setmainfont{Tribun ADF Std}

\title{Temporal dataflow graphs and their application to generative music}

\begin{document}
	\maketitle
	
	\section{Introduction}
	\section{Presentation}
	\subsection{Graph definitions}
	\subsubsection{Why multiple graphs ?}
	\subsubsection{Hierarchy}
	
	\subsection{Inner data and outer data}
	Inner data : reduces the possibilities
	
	Outer data : have to care about overwriting
	
	\subsection{Connection types}
	\subsubsection{Opportunistic}
	An execution of a sub-graph can happen whenever any of the nodes of a sub-graph are executing. 
	An order has to be provided.
	\subsubsection{Pessimistic}
	An execution of a sub-graph can only happen when all the nodes of the sub-graph are executing.
	\subsubsection{Delayed}
	A connection between an output and an input is delayed through bufferisation in a queue.
	
	The bufferisation can behave in two way :
	\begin{itemize}
		\item Readers of the buffer always start from the same point (the beginning of the previous function in the callback chain).
		\item Readers of the buffer continue from the latest read position.
		Question : what happens in case of multiple read ?
	\end{itemize}
	
	
	\subsubsection{Parallelized}
	All the sub-graphs are run with the same input environment. 
	The resulting values put in the environment are undefined, but it is guaranteed that all values returned will have been set.
	
	\subsubsection{Replaced}
	All the sub-graphs are run with the same input environment. 
	The resulting values put in the environment are those of the latest graph execution.
	
	\section{Meaning}
	Divide the time in abstract segmetn according to the activation graph; gives us a set of equations for each variable.
	\section{Implementation}
	\section{Examples}
\end{document}